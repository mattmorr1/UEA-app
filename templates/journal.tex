\documentclass[10pt,twocolumn]{article}

% Packages
\usepackage[utf8]{inputenc}
\usepackage{amsmath,amssymb,amsthm}
\usepackage{graphicx}
\usepackage{hyperref}
\usepackage[margin=0.75in]{geometry}
\usepackage{abstract}
\usepackage{fancyhdr}
\usepackage{natbib}
\usepackage{times}

% Header/Footer
\pagestyle{fancy}
\fancyhf{}
\fancyhead[L]{\footnotesize \textit{Journal Article}}
\fancyhead[R]{\footnotesize \thepage}
\renewcommand{\headrulewidth}{0.4pt}

% Abstract styling
\renewcommand{\abstractnamefont}{\normalfont\bfseries}
\renewcommand{\abstracttextfont}{\normalfont\small}

% Theorem environments
\newtheorem{theorem}{Theorem}
\newtheorem{lemma}[theorem]{Lemma}
\newtheorem{proposition}[theorem]{Proposition}
\newtheorem{corollary}[theorem]{Corollary}
\theoremstyle{definition}
\newtheorem{definition}{Definition}
\newtheorem{example}{Example}

\title{\textbf{Your Paper Title Here}\\[0.5em]
\large A Comprehensive Study}

\author{
Author One$^{1}$ \and Author Two$^{2}$\\[0.5em]
\small $^{1}$Institution One, City, Country\\
\small $^{2}$Institution Two, City, Country\\[0.3em]
\small \texttt{author1@example.com, author2@example.com}
}

\date{\today}

\begin{document}

\twocolumn[
  \begin{@twocolumnfalse}
    \maketitle
    \begin{abstract}
    This paper presents a comprehensive analysis of... The abstract should be a single paragraph of approximately 150-250 words that summarizes the key points of your research, including the problem statement, methodology, main findings, and conclusions.
    
    \noindent\textbf{Keywords:} keyword1, keyword2, keyword3, keyword4
    \end{abstract}
    \vspace{1em}
  \end{@twocolumnfalse}
]

\section{Introduction}

The introduction should provide context for your research and clearly state the problem you are addressing. Include a brief literature review and state your research objectives.

\subsection{Background}

Provide relevant background information here.

\subsection{Contributions}

The main contributions of this paper are:
\begin{enumerate}
    \item First contribution
    \item Second contribution
    \item Third contribution
\end{enumerate}

\section{Related Work}

Discuss prior work in the field and how your research builds upon or differs from existing approaches.

\section{Methodology}

\subsection{Problem Formulation}

Let $X$ denote the input space and $Y$ the output space. We define our problem as:

\begin{equation}
\min_{f \in \mathcal{F}} \mathbb{E}_{(x,y) \sim \mathcal{D}} [\ell(f(x), y)]
\label{eq:objective}
\end{equation}

where $\ell$ is the loss function and $\mathcal{D}$ is the data distribution.

\subsection{Proposed Approach}

Describe your methodology in detail.

\begin{theorem}
Statement of your main theoretical result.
\end{theorem}

\begin{proof}
Proof of the theorem goes here.
\end{proof}

\section{Experiments}

\subsection{Experimental Setup}

Describe your experimental setup, including datasets, baselines, and evaluation metrics.

\subsection{Results}

% Example table
\begin{table}[htbp]
\centering
\caption{Experimental Results}
\label{tab:results}
\begin{tabular}{|l|c|c|c|}
\hline
\textbf{Method} & \textbf{Metric 1} & \textbf{Metric 2} & \textbf{Metric 3} \\
\hline
Baseline 1 & 0.85 & 0.72 & 0.91 \\
Baseline 2 & 0.87 & 0.74 & 0.89 \\
\textbf{Ours} & \textbf{0.92} & \textbf{0.81} & \textbf{0.95} \\
\hline
\end{tabular}
\end{table}

As shown in Table~\ref{tab:results}, our method outperforms all baselines.

\section{Discussion}

Discuss the implications of your findings and any limitations of your approach.

\section{Conclusion}

Summarize your main findings and suggest directions for future work.

\section*{Acknowledgments}
We thank the reviewers for their helpful comments. This work was supported by...

\bibliographystyle{plainnat}
\bibliography{references}

\end{document}
